\begin{frame}{Concept: FE model calibration}
One use of FE models is to calibrate some unknown variable(s) - e.g. material parameters - to match known experimental data.

To do so, one need: 
\begin{itemize}
\item 1 or more FE models that are parametrised with the unknown variable(s) to calibrate;
\item a way to run FE models with parameters automatically;
\item a way to process FE models so that it outputs the value(s) of interest;
\item the corresponding experimental data;
\item a process to vary the parameters for the FE to match the experimental data.
\end{itemize}

\end{frame}


\begin{frame}{Concept: FE model calibration}
One use of FE models is to calibrate some unknown variable(s) - e.g. material parameters - to match known experimental data.

To do so, one need: 
\begin{itemize}
\item {\color{blue} 1 or more FE models that are parametrised with the unknown variable(s) to calibrate};
\item {\color{darkgreen} a way to run FE models with parameters automatically};
\item {\color{orange} a way to process FE models so that it outputs the value(s) of interest};
\item the corresponding experimental data;
\item {\color{darkgreen} a process to vary the parameters for the FE to match the experimental data}.
\end{itemize}

{\color{blue} Abaqus scripting interface}; {\color{darkgreen} \href{https://github.com/mengomarlene/opti4Abq}{opti4Abq} toolbox framework};

{\color{orange} e.g. \href{https://github.com/mengomarlene/postPro4Abq}{postPro4Abq} toolbox}
\end{frame}